Imagine your favorite college professor standing behind a podium in the center of Michigan Stadium in Ann Arbor, lecturing 109,000 students. Though that sounds like an unlikely scenario, Massive Open Online Courses, MOOCs, have practically made that a reality by offering previously exclusive classes to mass audiences. However, as the barriers to entry for MOOCs are very low, student dropout, referred to as student `stopout' \cite{breslow2013studying}, is very high. We believe that studying why students \sti will enable us to more fully understand how students learn in MOOCs.

This thesis applies a variety of machine learning algorithms to predict student persistence in MOOCs. We built predictive models by utilizing a framework that went through the following steps: organizing and curating the data, extracting predictive, sophisticated features, and developing a distributed, parallelizable framework. We built models capable of predicting stopout with AUCs\footnote{area under the curve of the receiver operating characteristic} of up to 0.95. These models even give an indication of whether students stopout because of predisposed motivations or due to course content. Additionally, we uncovered a number of findings about the factors indicative of stopout. These factors are presented in Chapter \ref{chap:conc}. Through the prediction framework we hope to help educators understand the factors of persistence in MOOCs and provide insight that prevents stopout. To our knowledge, this is the first in-depth, accurate prediction of stopout in Massive Open Online Courses.